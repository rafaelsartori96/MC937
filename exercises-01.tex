% Rafael Sartori M. dos Santos, 186154
\documentclass[english,a4paper]{article}

% Título
\title{MC937 -- Exercises 1}
\author{Rafael Sartori M. Santos, 186154}
\date{12 de março de 2020}

% Configuração do documento
\setlength{\parskip}{3pt} % espaço entre parágrafos
\usepackage{geometry} % alterar geometria do papel
\geometry{a4paper,left=2.5cm,right=2.5cm,top=2cm,bottom=2cm} % menor margem
\usepackage[utf8]{inputenc} % tipo de documento UTF-8
\usepackage{mathtools} % permitir expressões matemáticas
\usepackage{babel} % configuração da lingua portuguesa
\usepackage{caption} % para legenda de tabelas e figuras
\usepackage{cleveref} % para referenciar tabelas e figuras melhor
\usepackage{indentfirst} % indentação de todo primeiro parágrafo
\usepackage{verbatim} % pacote para incluir arquivos em verbatim
\usepackage{mdframed} % para enquadrar coisas
\usepackage{graphicx} % para produzir gráficos
\usepackage{subcaption} % para subfigures


% Início do documento
\begin{document}

\maketitle

\section*{Exercise 1}

  What are the areas related to Computer Graphics?  How do they relate?

  \begin{mdframed}
    \begin{small}
      Computer graphics is generally associated with image processing, computer vision, computational visualization.

      These areas are mixed in many applications and their usage is extensive. There are great problems to be solved in these areas and some will only be solved by mixing these areas, such as to build a safe autonomous car or a cashier-free supermarket.

      Image processing and computer graphics relate on the image itself. Analysing and correcting images by changing properties like contrast, brightness and even cropping must be considered by both areas. On computer graphics, these concepts must be used in order to create an accurate representation.

      Computer vision can, for instance, make use of computer graphics to recreate the scene of given image, simulating, predicting changes and/or recognizing patterns. This probably will make use of image processing to be done more efficiently.

      Computational visualization may only be possible using computer graphics because of the design of the graphics unit in computers today. Creating a representation of data will surely use computer graphics for the best user interaction experience, such as viewing data on a 3D space.
    \end{small}
  \end{mdframed}


\section*{Exercise 2}

  Is it correct to say that the Artificial Vision area depends on the Image Processing area? Justify your answer.

  \begin{mdframed}
    \begin{small}
      It is. Digital images realistically must be treated in order to recognize patterns and focusing different image components to discern objects. Using the image without processing it would be unecessary and too complex to be analyzed using the computational capacity that we have available today.

      Isolating information from the image brings down the computational capacity required to identify objects, making it efficient and, in some cases, very easily done by the computer. This is the case for factories that utilize computer vision to identify if the products are being produced without flaws.
    \end{small}
  \end{mdframed}


\section*{Exercise 3}

  What are the basic steps of a typical Artificial Vision system? Briefly describe each of these steps.

  \begin{mdframed}
    \begin{small}
      Artificial vision can be done by:
      \begin{itemize}
        \item \textbf{Acquiring images:} it is necessary to acquire the images to work with. Depending on the application (a moving computer body, a static camera, a sattelite, a drone), acquiring the image may be a problem to overcome because of limited resources like power draw or low bandwidth. It may not be the only resource that is being used to detect the space around the machine (it can use other kinds of sensors, such as distance sensor), but images can be used to recreate the most realistic space.
        \item \textbf{Recognizing the space:} image processing will be done here to help the machine isolate and identify objects with the help of the other resources acquired (multiple cameras, multiple sensors, other kinds of camera, such as infrared).
        \item \textbf{Recreating the space:} using isolated objects recoginized on the previous step, the machine can create a virtual space to help avoid collisions based on speed and acceleration, for example. This space is what we can call artificial vision, since it will be handled by the computer.
      \end{itemize}
    \end{small}
  \end{mdframed}


\section*{Exercise 4}

  What is the main difference between applications of Scientific Visualization and Information Visualization?  Give examples of each.

  \begin{mdframed}
    \begin{small}
      Information visualization differs from scientific visualization by what is shown on the visualization. Scientific visualization is a extensive topic, showing what ranges from planets to viruses, images created with the most advanced image processing on medicine, for example. Information visualization focuses on relations between kinds of data and connections, showing graphs or frequency graphs.
    \end{small}
  \end{mdframed}

\end{document}
